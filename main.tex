\documentclass[12pt]{article}

\usepackage{preamble}
\newcommand{\sgn}[1]{\text{sgn}(#1)}

\title{APS 2024 Abstract}
\author{\vspace{-5ex}}
\date{\vspace{-5ex}}

\begin{document}

\maketitle

\section*{Title: Horizon Tracking in SpECTRE with Task-Based Parallelism}

\subsubsection*{Authors: Kyle Nelli, Mark Scheel, Will Throwe, Geoffrey Lovelace}

A feature of the Generalized Harmonic formulation of Einstein's Equations (using the Damped Harmonic gauge condition), is that the physical singularities inside black holes are not dealt with by the gauge condition, as with moving punctures. Instead, they are removed by excising most of the interior of apparent horizons. However, to keep the excision regions within the horizons, horizon tracking is necessary. The Spectral Einstein Code (SpEC) and SpECTRE implement this horizon tracking using control theory to inform time-dependent coordinate mappings that dynamically adjust the mesh to keep the excision region inside the horizon. Unlike SpEC, SpECTRE utilizes task-based parallelism which adds subtle complications to the horizon tracking algorithm. Here I will present these complications and explain how SpECTRE's implementation of horizon tracking deals with them.

\end{document}